\section{Parallele Architekturen}

\begin{compactitem}
\item Architektur eines einzelnen Prozessors
\item Infrastruktur über mehrere Prozessoren hinweg
\end{compactitem}

\subsection{Single Chip}

Da die Performance der Taktraten eines Prozessors pro Jahr ungefähr um 30\%
steigt, die Performance von Applikationen aber um 55\% - 75\% pro Jahr steigt,
kann die Taktrate nicht alleine für den Speedup herangezogen werden.

\subsection{Fünf Ebenen von Parallelität}

\begin{compactitem}
\item auf Bitebene
\item Pipelining
\item auf der Ebene von Funktionalen Einheiten (ALU, FPU, etc.)
\item Prozessebene
\item Threadebene
\end{compactitem}

\subsubsection{Die Bitebene}

Mit der Bitebene sind die Algorithmen gemeint, welche z.B. bei den Carry-Look-Ahead
Addern benützt werden. Diese Effizienzsteigerungen werden durch Parallelität anstatt
durch Sequentielles Addieren erreicht. Grundsätzlich können alle Bits in einem 
(Register) Word parallelisiert werden. Da die maximale Größe der ansteuerbaren
Adressen fürs erste so hoch ist, dass man jetzt noch nicht daran denkt, diese
jemals auszuschöpfen, ist die Entwicklung bei 64 Bit Wörtern stehen geblieben.

\subsubsection{Pipelining}

Pipelining nützt die Tatsache aus, dass jede Instruktion auf einem Prozessor
mehrere Schritte hat. Fetch, Decode, Execute und Write-back. Traditionelle
Prozessoren haben diese Schritte Sequentiell bearbeitet. 
