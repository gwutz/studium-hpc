\documentclass[twocolumn, a4paper, 10pt, DIV12]{scrreprt}
\usepackage[T1]{fontenc}
\usepackage{lmodern}
\usepackage[ngerman]{babel}
\usepackage[utf8]{inputenc}
\usepackage{graphicx}
\usepackage{xcolor}
\usepackage{colortbl}
\usepackage{multirow}
\usepackage{hhline}

\newlength{\iconwidth}
\setlength{\iconwidth}{1cm}

\definecolor{boxheadcol}{gray}{.6}
\definecolor{boxcol}{gray}{.9}

\newenvironment{displaybox}[2]{%
  \begin{center}
    \setlength\arrayrulewidth{0.75pt}%
    \arrayrulecolor{white}%
    \renewcommand{\arraystretch}{1.3}%
    \begin{tabular}{p{\iconwidth}p{\linewidth-4\tabcolsep-\iconwidth}}
      \multirow{2}{*}{#2}&\cellcolor{boxheadcol}\textbf{\sffamily\color{white}#1} \\%
      \hhline{~-}%
      &\cellcolor{boxcol}%
}{%
      \\
    \end{tabular}
  \end{center}%
}

\newenvironment{Tipp}{%
\begin{displaybox}{Tipp}{\includegraphics[width=\iconwidth]{data/icon-tipp}}}%
{\end{displaybox}}

\newenvironment{Problem}{%
\begin{displaybox}{Problemstellung}{\includegraphics[width=\iconwidth]{data/icon-hinweis}}}%
{\end{displaybox}}

\newcommand{\code}[1]{\texttt{\textcolor{blue}{#1}}}


\title{High Performance Computing}

\begin{document}
\maketitle
\chapter*{Einführung}
Eine (informale) Definition von Parallelität: verschiedene Prozessoreinheiten 
(CPUs, ALUs, FPUs, etc.) arbeiten simultan (z. B. parallel) um einen gemeinsamen
Task zu lösen.

Warum wollen wir das?
\begin{itemize}
\item Manche Applikationen benötigen einen Speedup z. B. Klimavorhersagen,
        Windtunnel, Motorkonstruktion, Atomkraftwerktests oder Spiele
\end{itemize}

\section*{Herausforderungen von Parallelabarbeitung}
\begin{itemize}
\item Rießige Speicheranforderungen
\item Hohe Durchsatzanforderungen (z. B. Reiseplatformen)
\item Verteilte und Kooperative Internetanwendungen
\item Datenreplikation und Ausfallstrategien (Rechenclouds)
\item Heterogenität und Interoperabilität
\end{itemize}

\section*{Geschwindigkeit}
Geschwindigkeit bzw. Performance ist unser \textbf{Hauptgrund} um Parallele
Programme zu schreiben. Formal ist die \code{Geschwindigkeitssteigerung für p
Prozessoren} folgendermaßen definiert:

\[
S_p = \frac{T_1}{T_p}
\]

\begin{itemize}
    \item p ist die Anzahl der Prozessoren
    \item \(T_p\) ist die Laufzeit unserer Applikation auf p Prozessoren
    \item \(T_1\) ist die Sequenzielle Laufzeit der gleichen Applikation (Laufzeit auf einem Prozessor)
\end{itemize}

Demzufolge ist \(S_p\) der relative Laufzeitvorteil dem man mit p Prozessoren erreichen kann, verglichen
mit einer sequenziellen Implementierung.\\

Den Geschwindigkeitszuwachs, den man für p Prozessoren erwarten kann: Potenziell \code{p}.
Dazu ein Rechenbeispiel: Unsere Applikation benötigt sequenziell 10s zur Ausführung. Wenn wir
nun 2 Prozessoren nutzen, könnte das oberste Limit für die Ausführung 5s betragen. Daraus ergibt
sich ein Speedup von \( S_p = \frac{10s}{5s} = 2 \) was unserer eingesetzen Prozessoranzahl entspricht.

Wir werden später noch sehen, dass auch Geschwindigkeitssteigerungen über linearem Zuwachs möglich sind.
Diese werden entsprechend \textit{superlinear} genannt.

Für typische Software kann man den Speedup zwischen \( 1 \leq S_p \leq p \) vermuten. Manche Programme
können sogar einen Einbruch der Geschwindigkeit mit Parallel Computing erleben. Nicht alle Applikationen
sind für Parallelisierung geeignet. Auch können schlechte Implementierungen für einen schlechten
Performancezuwachs schuld sein, obwohl die Applikation ansonsten für Parallelisierung geeignet wäre.

\section*{Amdahl's Law (1967)}
Amdahl beschrieb 1967 bereits den Geschwindigkeitszuwachs durch Parallelisierung. Er zerlegt ein Programm
in zwei Teile, den vollständig sequentiellen Teil (\(\alpha\))und den vollständig Parallelisierbaren Teil (\(1-\alpha\)).

Daraus folgt die parallele Laufzeit auf p Prozessoren:
\[ T_p = \alpha T_1 + (1-\alpha)*\frac{T_1}{p} = T_1 * (\alpha + \frac{(1-\alpha)}{p})\]

Daraus wiederum kann der Speedup durch obige Formel berechnet werden:

\[ S_p = \frac{T_1}{T_p} = \frac{1}{\alpha+\frac{(1-\alpha)}{p}} \leq \frac{1}{\alpha} \]

Daraus folgt: Eine Applikation, die zu 90\% perfect parallelisiert werden kann, kann nur bis zum
Faktor 10 beschleunigt werden, unabhängig davon, wieviele Prozessoren benutzt werden. Damit
bildet \(1/\alpha\) die obere Schranke.

\section*{Gustafson's Law (1988)}
Amdahl nahm in seiner These an, dass der sequentielle Part mit der Problemgröße wächst. Gustafson
Ansatz war, dass der sequentielle Teil der meisten Applikationen eine konstante Laufzeit hat.

Sei \(T_1(n)\) die sequentielle Laufzeit für die Problemgröße n. So folgt
\[ T_1(n) = \tau + v(n)\]
wobei \(\tau\) der konstante sequentielle Teil und v perfekt parallelisierbar ist. Daraus folgt die
Parallele Laufzeit auf p Prozessoren

\[ T_p(n) = \tau + \frac{v(n)}{p}\]

Nach Gustafson ist damit eine Speedup

\[ S_p(n) = \frac{T_1(n)}{T_p(n)} = \frac{\tau + v(n)}{\tau + \frac{v(n)}{p}} = \frac{\frac{\tau}{v(n)} + 1}{\frac{\tau}{v(n)} + \frac{1}{p}} \]

Wenn man nun die Problemgröße n gegen Unendlich gehen lässt, so wird die Laufzeit v(n) immer kleiner.

\[ \lim_{n \rightarrow \infty} S_p(n) = p\]

Solche Applikationen nennt man \textbf{Skalierbar}.

\end{document}
